\documentclass[10pt,a4paper]{article}
\usepackage[utf8]{inputenc}
\usepackage[francais]{babel}
\usepackage[T1]{fontenc}
\usepackage{amsmath}
\usepackage{amsfonts}
\usepackage{amssymb}
\author{Paul CLAVIER}
\title{[Info 524]Système d'exploitation}
\begin{document}
%head
\maketitle
\newpage
\tableofcontents
\newpage
%body
\section{Processus}
	\subsection{C'est quoi?}
		\begin{center}
			processus $\neq$ programme
		\end{center}
		Un processus est un programme en cours d’exécution. En particulier, un programme peut donner plusieurs processus. Avec un processus, l'OS garde de nombreuses informations comme:
		\begin{itemize}
			\item adresse mémoire de l’exécutable
			\item l'utilisateur qui a lancé le programme
			\item l'heure de début
			\item le PID
			\item la priorité
			\item le compteur ordinal
			\item la liste des fichiers ouverts
			\item l'état
			\item la liste des processus créés			
		\end{itemize}
	\subsection{États}
		Un processus peut être:
		\begin{itemize}
			\item en exécution
			\item bloqué
			\item en attente
		\end{itemize}
	\subsection{Création d'un processus}
		\begin{itemize}
			\item Sous POSIX, un seule manière de créer un processus: la fonction \emph{fork()}. Elle duplique le processus en cours. Le processus créé est identique au processus de départ. La seule différence est la valeur de retour de la fonction \emph{fork()}:\begin{itemize}
				\item dans le processus initial, la valeur est le PID du processus créé.
				\item dans le processus créé, fork renvoie 0
			\end{itemize}

			\item execl(chemin, args): remplace le processus par un nouveau processus en utilisant l’exécutable du chemin.
		\end{itemize}
		Les processus POSIX sont mis dans une structure arborescente. Chaque processus créé (avec fork) est un fils du processus qui le créé.
	\subsection{Mort d'un processus POSIX}
		La fonction kill() permet de tuer un processus. Lorsqu'un processus reçoit le signal SIG.TERM, il s’arrête. L'OS le supprime de la table des processus. Lorsqu'un processus s'arrête, ses sous-processus deviennent fils de init.
		\begin{description}
			\item[Remarque]: quand un processus s'arrête, le système garde de l'information pour pouvoir informer le processus père (fonction wait). Tant que le père n'a pas fait de "wait", le processus est conservé. On parle de zombie.
		\end{description}
	\subsection{Ordonnancement}
		\begin{description}
			\item[ordonnanceur]: partie de l'OS qui décide quel processus s'exécute. C'est un problème compliqué.
		\end{description}
		\subsubsection{non préemptif}
			L'OS n’arrete pas un processus en cours d"exécution.
		\subsubsection{préemptif}
			L'OS peut stopper un processus en exécution.
			\begin{itemize}
				\item Tourniquet avec quantum de temps: Les processus sont dans une file, et à intervalle de temps régulier, on remplace le processus en exécution par le suivant.
				\item Tourniquet avec priorité: chaque processus a une priorité. On exécute le premier processus le plus prioritaire pendant un intervalle de temps. En pratique, chaque priorité a sa propre file.
				\item Loterie: Chaque processus (pret) reçoit un nombre de tickets proportionnel à l'utilisation voulue du processeur. Les processus d'un même utilisateur se partagent les tickets, les processus plus prioritaires reçoivent plus de tickets. L’ordonnanceur choisit un processus en tirant un ticket au hasard.
			\end{itemize}
\section{La mémoire}
		\subsection{espace d'adressage: abstraction de la mémoire}
			Avec un système multi-programmé, l'allocation naïve ne marche pas bien. La mémoire se fragmente au fur et a mesure de l'apparition des processus. On peut faire croire au processus que sa mémoire commence à 0.
		\subsection{Représentation de la mémoire dans l'OS}
			L'OS connaît le quantité de RAM à sa disposition, il doit gérer les parties libres/occupées de la mémoire.
			\begin{itemize}
				\item Avec un tableau de bits: chaque bit représente l'état d'un bloc mémoire.
				\item Avec une liste chaînée de descripteurs de zones libres.
			\end{itemize}
		\subsection{Algorithme d'affectation}
			Il faut choisir une zone libre de mémoire à la demande:
			\begin{itemize}
				\item First Fit: On choisit le premier emplacement libre assez grand.
				\item Next Fit: On prends le premier emplacement assez grand en partant de l'allocation précédente.
				\item Best Fit: On prends le meilleur emplacement (plus proche possible).
				\item Worst Fit: On prends le pire.
			\end{itemize}
		\subsection{Mémoire virtuelle, pagination}
			Objectif: pouvoir ajouter de la mémoire à la demande des processus. On découpe la RAM en blocs de taille fixe (4 ko). Ces blocs sont appelés \emph{cadres}. L'espace d'adressage va de 0x000000... à 0xfffffff... (toutes les adresses disponibles). Cet espace virtuel est découpé en blocs de taille fixe (même que celle des cadres), appelés pages. Les informations sur les pages sont stockées dans la table des pages.
		\subsection{Table des pages}
			Pour chaque page, on stocke:
			\begin{itemize}
				\item \no de cadre correspondant
				\item boolean pour savoir si la page est in (RAM) ou out (swap)
				\item boolean pour savoir si la page a été lue (référence)
				\item boolean pour savoir si la page a été modifiée (dirty)
			\end{itemize}
			La traduction est faite par la MMU en utilisant la table des pages (niveau hardware).
		\subsection{Pagination et swaping}
			Pour les systèmes avec peu de mémoire, on utilise en général un morceau du disque dur comme mémoire secondaire. Cette mémoire ne peut pas remplacer la RAM. Elle sert à stocker des cadres qui ne sont pas utilisés. Lorsque la RAM est pleine, la MMU ne peut pas donner de nouveaux cadres, mais le noyau peut libérer des cadres en les copiant sur le disque. Le choix des cadres peut être fait de plusieurs manières. On parle d'algorithmes de remplacement de pages.
			\begin{itemize}
				\item FIFO: premier cadre arrivé, premier cadre supprimé.
				\item LRU: "least recently used"
				\item NRU: "not recently used".
			\end{itemize}
	\section{Système de fichier}
		Un fichier est la plus petite unité d'information du disque accessible au processus.
		\paragraph{Hypothèse}: On suppose que tous les supports ont deux opérations:\begin{itemize}
			\item read(k): lire l'octet k du support
			\item write(k, o): écrire l'octet o à l'emplacement k
		\end{itemize}
		\paragraph{} Il faut définir les fichiers uniquement à partir de ces fonctions.
		\subsection{abstraction des fichiers}
			identification des fichiers: chaque fichier à un nom avec une ou plusieurs extensions, des droits associés aux utilisateurs, des attributs.
		\subsection{accès aux fichiers}
			se fait en utilisant des fonctions "haut niveau" qui n'utilise que les deux fonctions read et write. On peut ensuite programmer des fonctions plus complexes à partir de ça.
		\subsection{organisation des fichiers}
			\begin{itemize}
				\item en vrac: tous les fichiers au même endroit
				\item en arbre: chaque répertoire contient des fichier et d'autres répertoires. Chaque fichier est identifié par la branche de répertoires et son nom (chemin).
				\item en base de donnée
			\end{itemize}
		\subsection{implantation des systèmes de fichiers}
			\subsubsection{Partition}		
				Un partition est une zone consécutive du disque.
			\subsubsection{Organisation des fichiers sur la partition}
				\begin{itemize}
					\item les fichiers sont stockés les uns à la suite des autres
					\item table de blocs en mémoire (FAT)
					\item les fichiers sont représentés par des inodes
				\end{itemize}
		
\end{document}